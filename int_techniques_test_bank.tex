\documentclass[10pt]{exam}
%\usepackage[pdf]{pstricks}
%\usepackage{pstricks-add}
%\usepackage{pst-plot,pst-math}
\usepackage{enumerate}
\usepackage{pgfplots}
\usepackage{graphicx, amsmath, amssymb}
\usepgfplotslibrary{polar}

\title{Integration Techniques Test Bank}
\date{}



\tikzset{
  jumpdot/.style={mark=*,solid},
  excl/.append style={jumpdot,fill=white},
  incl/.append style={jumpdot,fill=black},
}


\begin{document}
\maketitle

\begin{questions}

\question % ans = D
Compute the antiderivative using integration by parts: $ \int x \cos(x) dx$


\begin{choices}
\choice $x \cos(x) + \sin(x)$
\choice $\frac{x^2}{2}\sin(x)$
\choice $x \cos(x) - \sin(x)$
\choice $x \sin(x) + \cos(x)$
\choice $x (\sin(x) + \cos(x))$
\end{choices}


\question Compute the antiderivative \label{trigsub}

$$\int \frac{2}{x^3\sqrt{x^2 -1}}dx$$
assuming $x>1$.

\begin{choices}



\choice $\sec^{-1}(x) + \frac{\sqrt{x^2-1}}{x}$	

\choice $\tan^{-1}(x) + \frac{\sqrt{x^2-1}}{x^2}$

\choice $\sec^{-1}(x) + \frac{\sqrt{x^2-1}}{x^2}$ %correct

\choice $\frac{\sec^{-1}(x)}{2} + \frac{\sqrt{x^2-1}}{2x^2}$

\choice $\sec^{-1}(x) - \frac{\sqrt{x^2-1}}{x}$	

\end{choices}

\question \label{B} True or false: an \textbf{improper} integral is an integral that does not exist or is $\infty$ or $-\infty$. % false

\fullwidth{Questions \ref{A}  and \ref{C} are based on the following integrals:}



\begin{enumerate}[I]
\item $\int_{0}^{1}\frac{1}{\sqrt{x}} dx$
\item $\int_{1}^{2}\frac{1}{x} dx$
\item $\int_{-2\pi}^{2\pi}\sin(x) dx$
\item $\int_{0}^{\infty}\sin(x) dx$
\end{enumerate}

\question \label{A} Which of these integrals is an \textbf{improper} Riemann integral?

\begin{choices} % B
\choice I, II and IV
\choice I and IV
\choice none are improper
\choice all are improper
\choice IV only
\end{choices}





\question \label{C} Which of those integrals is convergent and finite (meaning the integral exists and is a finite number, not $-\infty$ or $\infty$)?

\begin{choices} % E
\choice II and III
\choice I only
\choice all are convergent and finite
\choice IV only
\choice I, II, and III
\end{choices}


\question 
Perform a partial fraction decomposition on the rational function
$$\frac{3x+1}{(x+3)(x-9)}$$
and chose the appropriate equation below.

\begin{choices} % E
\choice $\frac{3x+1}{(x+3)(x-9)} = \frac{A}{x+3}+ \frac{B}{x-9}$

\choice $\frac{3x+1}{(x+3)(x-9)} = \frac{7/3}{x+3}+ \frac{2/3}{x-9}$
\choice $\frac{3x+1}{(x+3)(x-9)} = \frac{1/12}{x+3}+ \frac{-1/12}{x-9}$
\choice $\frac{3x+1}{(x+3)(x-9)} = \frac{-1/12}{x+3}+ \frac{1/12}{x-9}$
\choice $\frac{3x+1}{(x+3)(x-9)} = \frac{2/3}{x+3}+ \frac{7/3}{x-9}$
\end{choices}


\question

Given the partial fraction decomposition
$$\frac{2x-1}{(2x+6)(3x-1)} = \frac{7/10}{(2x+6)} - \frac{1/20}{(3x-1)}$$
compute the antiderivative
$$\int \frac{2x-1}{(x+6)(x-1)} dx.$$

\begin{choices}

\choice $7/20 \ln|2x+6| -1/60 \ln|3x-1|$ % correct

\choice $7/10 \ln|2x+6| -1/20 \ln|3x-1|$

\choice $10/7 \ln|2x+6| -20 \ln|3x-1|$
\choice $14/10 \ln|2x+6| -3/20 \ln|3x-1|$
\choice $14 \ln|2x+6| - \ln|3x-1|$


	
\end{choices}




\question Compute the antiderivative $\int \cos^2(x) \sin^3(x) dx$.


\begin{choices} % D

\choice $\cos^4(x) - \cos^2(x)$
\choice $ -\frac{\cos^5(x)}{5} + \frac{\cos^3(x)}{3}$
\choice $\frac{\cos^3(x)}{3}\frac{\sin^4(x)}{4}$
\choice $ \frac{\cos^5(x)}{5} - \frac{\cos^3(x)}{3}$
	
\choice $\sin^4(x) - \sin^2(x)$
\end{choices}


\question \label{parts} Compute the antiderivative $\int \sin^{-1}(x)dx$.

\begin{choices} % E

\choice $-\sqrt{1-x^2} + x\sin^{-1}(x)$
\choice $-\cos^{-1}(x)$
\choice $-\sqrt{1+x^2} + x\sin^{-1}(x)$
\choice $\sqrt{1+x^2} - x\sin^{-1}(x)$
	\choice $\sqrt{1-x^2} + x\sin^{-1}(x)$
\end{choices}


\question Choose the answer that best describes the integral 
$\int_0^\infty \sin(x) dx$.

\begin{choices} % D
\choice The integral equals 0 because the positive area cancels with the negative area.

\choice The integral  is $\infty$.

\choice The integral  is some finite number between 1 and -1.

\choice The integral is divergent, meaning the limit does not exist.

\choice The integral oscillates between 1 and -1.


	
\end{choices}


\question Compute the limit $$\lim_{x\to 0^+} \frac{x}{\ln(x)}.$$

\begin{choices}
\choice $\infty$
\choice $0$
\choice $1$
\choice does not exist
\choice $-\infty$
	
\end{choices}




\question True or false: $e^x$ grows faster than $x^{10}$ as $x \to \infty$.


\question Compute the antiderivative 

$$\int \frac{1}{x^2-6x+14}dx$$

\begin{choices} % A
\choice $\frac{\sqrt{5}}{5} \arctan(\frac{x-3}{\sqrt{5}})$

\choice $\frac{\sqrt{3}}{3} \arctan(\frac{x-7}{\sqrt{3}})$

\choice $\frac{1}{5}\arctan(\frac{x-3}{5})$
\choice $\frac{1}{5}\arctan(\frac{x-3}{\sqrt{5}})$

\choice $\frac{1}{3}\arctan(\frac{x-7}{\sqrt{3}})$


	
\end{choices}



\question True or false: integration is the same thing as anti-differentiating.  


\question Let $f(x) =  2x^2$.  From the choices given, choose the number $M$ that is the smallest bound on $|f'(x)|$ in the interval $[0, 5]$, or indicate that $f'(x)$ is not bounded if it is not bounded.

This means choose the number $M$ so that 

$$|f'(x)|\leq M$$

on the interval.

\begin{choices}
\choice $f'(x)$ is not bounded on the interval.
\choice 4
\choice 50
\choice 20
\choice 1
	
\end{choices}

\question True or false: in typical real world problems, we can approximate an integral using a computer and then see how close we were by computing it exactly.


\question \label{errbndq1} Say we want to approximate $\int_0^4 f(x)dx$ using Simpsons rule and we require an approximation that is within $\frac{1}{100}$ of the exact answer.  That is, we require

$$|\text{error}| = \left| \text{exact answer} - \text{our approximation}\right|\leq \frac{1}{100}.$$

An error bound for Simpsons rule is 
$$|\text{error}| \leq \frac{M(b-a)^5}{180n^4}$$
where $n$ is the number of subintervals.  \textbf{You may assume $M = 3$.}

From the choices below, choose the smallest number of subintervals $n$ that will give us the desired accuracy using this error bound.


\begin{choices}
\choice approximately 6.4
\choice 20
\choice 2
\choice approximately 10.3	
\choice 8
\end{choices}




\question Imagine that the Question \ref{errbndq1} is a real world application.  What is the result if you use \textbf{more} subintervals than the smallest number required?

\begin{choices}
\choice the approximation will not be accurate enough;
\choice there is no way to know whether the approximation will be accurate enough or not;
\choice the approximation will be accurate enough, but we have used more computer time than necessary;
\choice the approximation is even more accurate and nothing is lost;
\choice the approximate error will be smaller than the exact error;
\end{choices}
 



\end{questions}

\end{document}